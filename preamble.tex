\documentclass[abstract, english]{scrartcl}
\addtokomafont{disposition}{\boldmath}

% LuaLatex
% \usepackage{polyglossia}
% \setmainlanguage{english}

% \usepackage{fontspec}
% \setmainfont[
%     Ligatures=TeX,
%     SmallCapsFont={Latin Modern Roman Caps},
%     SlantedFont={* Slanted},
%     ItalicFeatures  = {
%         SmallCapsFont = {LMRomanCaps10-Oblique}
%     },
%     ]{Latin Modern Roman}

% pdflatex
\usepackage[english]{babel}
\usepackage[T1]{fontenc}
\usepackage[utf8]{inputenc}
\usepackage{lmodern}

\usepackage{microtype}

\usepackage{todonotes}
\usepackage{blindtext}

\usepackage{amsmath}
\usepackage{amssymb}
\usepackage{mathdots}
\usepackage{mathtools}
\usepackage{csquotes}
\usepackage{enumitem}

\usepackage{subcaption}
\usepackage[font=small, labelfont=bf, format=hang, indention=-2em]{caption}

\usepackage{algorithm}
\usepackage[noend]{algpseudocode}

\usepackage[hyperref,thmmarks,amsmath]{ntheorem}

\usepackage{url}
\usepackage[pdftex, hypertexnames=false, unicode]{hyperref}
\usepackage[nameinlink, noabbrev]{cleveref}

\usepackage{tikz}
\usepackage{pgfplots}
\pgfplotsset{compat=1.12}
\usetikzlibrary{calc}

%%%

\theoremstyle{plain}
\theoremheaderfont{\rmfamily\bfseries\upshape\boldmath}
\theorembodyfont{\itshape}
\theoremseparator{}
\theoremsymbol{}
\newtheorem{definition}{Definition}
\newtheorem{theorem}[definition]{Theorem}
\newtheorem{lemma}[definition]{Lemma}
\newtheorem{problem}[definition]{Problem}
\newtheorem{observation}[definition]{Observation}

\theoremstyle{plain}
\theoremheaderfont{\scshape}
\theorembodyfont{}
\theoremseparator{}
\theoremsymbol{\qed}
\newtheorem*{proof}{Proof}

\theoremstyle{break}
\theoremheaderfont{\rmfamily\bfseries\upshape\boldmath}
\theorembodyfont{}
\theoremseparator{}
\theoremsymbol{}
\newtheorem{Algorithm}[definition]{Algorithm}

\crefname{observation}{observation}{observations}
\Crefname{observation}{Observation}{Observations}
\crefname{Algorithm}{algorithm}{algorithms}
\Crefname{Algorithm}{Algorithm}{Algorithms}

%%%

\newcommand{\R}{\mathbb{R}}
\newcommand{\Ha}{\mathcal{H}}
\newcommand{\F}{\mathcal{F}}
\newcommand{\Ge}{\mathcal{G}}
\newcommand{\Oh}{\mathcal{O}}

\DeclareMathOperator{\dist}{d}
\DeclareMathOperator{\pow}{pow}
\DeclareMathOperator{\PD}{PD}
\DeclareMathOperator{\cell}{cell}
\DeclareMathOperator{\chor}{chor}
\DeclareMathOperator{\aff}{aff}
\DeclareMathOperator{\CH}{CH}
\DeclareMathOperator{\CHb}{CH_b}
\DeclareMathOperator{\CHt}{CH_t}
\DeclareMathOperator{\IL}{IL}
\DeclareMathOperator{\ls}{ls}
\DeclareMathOperator{\Time}{T}

\newcommand{\abs}[1]{\left\vert #1 \right\vert}
\newcommand{\norm}[1]{\left\Vert #1 \right\Vert}
\newcommand{\scalar}[1]{\left\langle #1 \right\rangle}
\newcommand{\powerset}[1]{\mathcal{P}(#1)}

\newcommand{\define}[1]{\emph{#1}}

\newcommand{\Wlog}{w.\,l.\,o.\,g.\ }
\newcommand{\WLOG}{W.\,l.\,o.\,g.\ }

% From llncs.cls
\def\squareforqed{\hbox{\rlap{$\sqcap$}$\sqcup$}}
\def\qed{\ifmmode\squareforqed\else{\unskip\nobreak\hfil \penalty50\hskip1em\null\nobreak\hfil\squareforqed \parfillskip=0pt\finalhyphendemerits=0\endgraf}\fi}

% See https://tex.stackexchange.com/questions/4302/prettiest-way-to-typeset-c-cplusplus
\def\CC{{C\nolinebreak[4]\hspace{-.05em}\raisebox{.4ex}{\tiny\textbf{++}}}\xspace}
