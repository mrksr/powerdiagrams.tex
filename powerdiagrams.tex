\documentclass[abstract, english]{scrartcl}
\addtokomafont{disposition}{\boldmath}

% LuaLatex
% \usepackage{polyglossia}
% \setmainlanguage{english}

% \usepackage{fontspec}
% \setmainfont[
%     Ligatures=TeX,
%     SmallCapsFont={Latin Modern Roman Caps},
%     SlantedFont={* Slanted},
%     ItalicFeatures  = {
%         SmallCapsFont = {LMRomanCaps10-Oblique}
%     },
%     ]{Latin Modern Roman}

% pdflatex
\usepackage[english]{babel}
\usepackage[T1]{fontenc}
\usepackage[utf8]{inputenc}
\usepackage{lmodern}

\usepackage{microtype}

\usepackage{todonotes}
\usepackage{blindtext}

\usepackage{amsmath}
\usepackage{amssymb}
\usepackage{mathdots}
\usepackage{mathtools}
\usepackage{csquotes}
\usepackage{enumitem}

\usepackage{subcaption}
\usepackage[font=small, labelfont=bf, format=hang, indention=-2em]{caption}

\usepackage{algorithm}
\usepackage[noend]{algpseudocode}

\usepackage[hyperref,thmmarks,amsmath]{ntheorem}

\usepackage{url}
\usepackage[pdftex, hypertexnames=false, unicode]{hyperref}
\usepackage[nameinlink, noabbrev]{cleveref}

\usepackage{tikz}
\usepackage{pgfplots}
\pgfplotsset{compat=1.12}
\usetikzlibrary{calc}

%%%

\theoremstyle{plain}
\theoremheaderfont{\rmfamily\bfseries\upshape\boldmath}
\theorembodyfont{\itshape}
\theoremseparator{}
\theoremsymbol{}
\newtheorem{definition}{Definition}
\newtheorem{theorem}[definition]{Theorem}
\newtheorem{lemma}[definition]{Lemma}
\newtheorem{problem}[definition]{Problem}
\newtheorem{observation}[definition]{Observation}

\theoremstyle{plain}
\theoremheaderfont{\scshape}
\theorembodyfont{}
\theoremseparator{}
\theoremsymbol{\qed}
\newtheorem*{proof}{Proof}

\theoremstyle{break}
\theoremheaderfont{\rmfamily\bfseries\upshape\boldmath}
\theorembodyfont{}
\theoremseparator{}
\theoremsymbol{}
\newtheorem{Algorithm}[definition]{Algorithm}

\crefname{observation}{observation}{observations}
\Crefname{observation}{Observation}{Observations}
\crefname{Algorithm}{algorithm}{algorithms}
\Crefname{Algorithm}{Algorithm}{Algorithms}

%%%

\newcommand{\R}{\mathbb{R}}
\newcommand{\Ha}{\mathcal{H}}
\newcommand{\F}{\mathcal{F}}
\newcommand{\Ge}{\mathcal{G}}
\newcommand{\Oh}{\mathcal{O}}

\DeclareMathOperator{\dist}{d}
\DeclareMathOperator{\pow}{pow}
\DeclareMathOperator{\PD}{PD}
\DeclareMathOperator{\cell}{cell}
\DeclareMathOperator{\chor}{chor}
\DeclareMathOperator{\aff}{aff}
\DeclareMathOperator{\CH}{CH}
\DeclareMathOperator{\CHb}{CH_b}
\DeclareMathOperator{\CHt}{CH_t}
\DeclareMathOperator{\IL}{IL}
\DeclareMathOperator{\ls}{ls}
\DeclareMathOperator{\Time}{T}

\newcommand{\abs}[1]{\left\vert #1 \right\vert}
\newcommand{\norm}[1]{\left\Vert #1 \right\Vert}
\newcommand{\scalar}[1]{\left\langle #1 \right\rangle}
\newcommand{\powerset}[1]{\mathcal{P}(#1)}

\newcommand{\define}[1]{\emph{#1}}

\newcommand{\Wlog}{w.\,l.\,o.\,g.\ }
\newcommand{\WLOG}{W.\,l.\,o.\,g.\ }

% From llncs.cls
\def\squareforqed{\hbox{\rlap{$\sqcap$}$\sqcup$}}
\def\qed{\ifmmode\squareforqed\else{\unskip\nobreak\hfil \penalty50\hskip1em\null\nobreak\hfil\squareforqed \parfillskip=0pt\finalhyphendemerits=0\endgraf}\fi}

% See https://tex.stackexchange.com/questions/4302/prettiest-way-to-typeset-c-cplusplus
\def\CC{{C\nolinebreak[4]\hspace{-.05em}\raisebox{.4ex}{\tiny\textbf{++}}}\xspace}

\xdefinecolor{tumblue}     {RGB}{  0,101,189}
\xdefinecolor{tumgreen}    {RGB}{162,173,  0}
\xdefinecolor{tumred}      {RGB}{229, 52, 24}
\xdefinecolor{tumivory}    {RGB}{218,215,203}
\xdefinecolor{tumorange}   {RGB}{227,114, 34}
\xdefinecolor{tumlightblue}{RGB}{152,198,234}

\tikzstyle{every node} = [font=\normalsize]

\tikzstyle{edge} = [very thick]
\tikzstyle{internal edge} = [edge, tumorange]
\tikzstyle{extremal edge} = [edge, tumgreen]
\tikzstyle{node on edge} = [fill=white, circle, inner sep=1pt, font=\small]
\tikzstyle{internal spheres} = [node on edge]
\tikzstyle{extremal spheres} = [node on edge, very near start]

\tikzstyle{sphere center} = [tumred]
\tikzstyle{sphere radius} = [thin, scale=0.5]
\tikzstyle{point} = [circle, draw, fill, minimum size=4pt, inner sep=0pt]


\title{Calculating Power Diagrams Using Convex Hulls}
\subtitle{Interdisciplinary Project}
\author{Markus Kaiser}
\institute{Technische Universität München\\
    \email{\href{mailto:markus.kaiser@in.tum.de}{markus.kaiser@in.tum.de}}}

\begin{document}

\maketitle
\begin{abstract}
    Cool Paper!
\end{abstract}

\section{Definition of Power Diagrams}
\label{sec:powerdiagrams}
\subsection{Diagrams}
\label{sub:diagramme}
\todo[inline]{Voronoi Diagrams}
\todo[inline]{Definition of Metric}
\todo[inline]{Power Diagrams as "Voronoi with Spheres"}
\subsection{Geometry}
\label{sub:geometry}
\todo[inline]{Faces, Facets, $\ldots$}
\todo[inline]{Polyhedra}
\todo[inline]{Dual Polyhedra}
\subsection{Observations}
\label{sub:observations}
\todo[inline]{Chordales}
\todo[inline]{Weird Consequences like Sphere not in it's cell}
\todo[inline]{Upper bounds for faces (special case 2D)}

\section{Embedding in $d+1$ Dimensions}
\label{sec:embedding_in_d_1_dimensions}
\subsection{The transform to $d+1$}
\label{sub:the_transform}
\todo[inline]{Transformation function $\Pi$}
\todo[inline]{Observation about distances}
\todo[inline]{Chordales are vertial projections}
\todo[inline]{Existance of 0-faces}

\subsection{Duality of points and hyperplanes in $E^{d+1}$}
\label{sub:duality_of_points_and_hyperplanes_in_e_d_1_}
\todo[inline]{Polarity function $\Delta$}
\todo[inline]{Observation about relative positioning}
\todo[inline]{Lemma 5 about the mapping $\gamma$?}

\subsection{Power Diagrams and Polyhedra}
\label{sub:power_diagrams_and_polyhedra}
\todo[inline]{Main Theorem: For any $(d+1)$ intersection of upper halfspaces, there exists an affinely equivalent power diagram in $d$ dimensions in $h_0$.}
\todo[inline]{A cell complex in $E^d$ is polytopal iff $C$ is a power diagram. (Why?)}
\todo[inline]{For any finite $M \subset E^{d+1}$, there exists a set of spheres such that $PD(S)$ is dual to $CH_b(M)$ and vice versa.}

\section{Implementation}
\subsection{Naive Algorithm}
\label{sub:naive_algorithm}
\todo[inline]{Combinatorical trys with brute force}

\subsection{The Algorithm}
\label{sub:the_algorithm}
\todo[inline]{3-Step-Algorithm}
\todo[inline]{Running Time (fairly trivial)}
\todo[inline]{Convex Hulls?}

\subsection{Incidence Lattices}
\label{sub:incidence_lattices}
\todo[inline]{Model as "Bidirectional Graph"}
\todo[inline]{Adding Faces to the Lattice is fun - Lubs and Groups}

\appendix
\section{Building the Code}
\label{sec:building_the_code}
\todo[inline]{Libraries}
\todo[inline]{Make, CMake}
\todo[inline]{Windows...}
\end{document}
